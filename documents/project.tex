\markright{Customising}
\section{Customising}
\label{Customising}
This work, as stated in the abstract, aims to implement a native daemon (constantly running) which will be exposed as a System Service to the Java application layer. The Java application will be able to communicate with the daemon through JNI and the native library, and the other way around will be implemented by using a callback function.\\
Since this project is focused on understanding the basics of Android, to afterwards integrate a native RTRM application as a System Service - therefore accessible by the Java environment - I decided to implement this simple daemon as a dummy-RTRM which will communicate with the application the number of available "cores" (actually will be just an \texttt{int}) and allow it to "require" or "release" cores (with a sort of \texttt{core++} and \texttt{core--} functions).
\subsection{Setting up the directory structure}
The followed approach has been to keep separate the main code from our customised code, so that our platform-specific code will be compiled when the user will choose to build our specific system, instead of the default one. This choice results in a more atomic and modular way to operate.\\
In the \texttt{aosp} directory, there's a folder named \texttt{device}, which typically contains folders indicating manifacturers, and into each manifacturer's folder, the specific architecture folder: in our case, I named \texttt{bosp} the manifacturer folder, and \texttt{p2012} the architecture folder. To properly organize files, we have a main \texttt{p2012} folder, and a \texttt{p2012-common} folder, which contains the largest part of our customised code.
\begin{verbatim}
device/
├── bosp/
│   ├── p2012/
│   │   ├── app/
│   ├── p2012-common/
│   │   ├── app/
│   │   ├── bin/
│   │   ├── framework/
│   │   └── lib/
\end{verbatim}
Let's briefly see what these folders will contain:
\begin{itemize}
	\item \texttt{p2012/app}: contains configuration files to identify the board, the architecture, the packages that will be built-in with this version, and the possible "Activities" that will be coded especially for this release.
	\item \texttt{p2012-common/app}: contains \texttt{ServiceApp} that will register our new service to the \textit{Service Manager} as a remote service, through the call
	\begin{verbatim}
		ServiceManager.addService(REMOTE_SERVICE_NAME, this.serviceImpl);
	\end{verbatim}
	\item \texttt{p2012-common/bin}: contains the C/C++ code of the \textit{native daemon}. Once exported as local module, its name will be added to the \texttt{init.rc} file, to let it start at bootup stage.
\end{itemize}