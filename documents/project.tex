\markright{Customising}
\section{Customising}
\label{Customising}
This work, as stated in the abstract, aims to implement a native daemon (constantly running) which will be exposed as a System Service to the Java application layer. The Java application will be able to communicate with the daemon through JNI and the native library, and the other way around will be implemented by using a callback function.\\
Since this project is focused on understanding the basics of Android, to afterwards integrate a native RTRM application as a System Service - therefore accessible by the Java environment - I decided to implement this simple daemon as a dummy-RTRM which will communicate with the application the number of available "cores" (actually will be just an \texttt{int}) and allow it to "require" or "release" cores (with a sort of \texttt{core++} and \texttt{core--} functions).
\subsection{Setting up the directory structure}
The followed approach has been to keep separate the main code from our customised code, so that our platform-specific code will be compiled when the user will choose to build our specific system, instead of the default one. This choice results in a more atomic and modular way to operate.\\
In the \texttt{aosp} directory, there's a folder named \texttt{device}, which typically contains folders indicating manifacturers, and into each manifacturer's folder, the specific architecture folder: in our case, the manifacturer folder has been named \texttt{bosp}, and the architecture folder \texttt{p2012}. To properly organize files, we have a main \texttt{p2012} folder, and a \texttt{p2012-common} folder, which contains the largest part of our customised code.
\begin{verbatim}
device/
├── bosp/
│   ├── p2012/
│   │   ├── app/
│   ├── p2012-common/
│   │   ├── app/
│   │   ├── bin/
│   │   ├── framework/
│   │   └── lib/
\end{verbatim}
Let's briefly see what these folders contain:
\begin{itemize}
	\item \texttt{p2012/app}: contains configuration files to identify the board, the architecture, the packages that will be built-in with this version, and the possible "Activities" that will be coded especially for this release.
	\item \texttt{p2012-common/app}: contains the Java code of the \texttt{ServiceApp} that will register our new service to the \textit{Service Manager} as a remote service, through the call
	\begin{verbatim}
		ServiceManager.addService(REMOTE_SERVICE_NAME, this.serviceImpl);
	\end{verbatim}
	\item \texttt{p2012-common/bin}: contains the C/C++ code of the \textit{native daemon}. Once exported as local module, its name will be added to the \texttt{init.rc} file, to let it start at bootup stage.
\definecolor{lightgray}{gray}{0.93}
  \vskip\baselineskip%
  \par\noindent\colorbox{lightgray}{%
    \begin{minipage}{0.953\textwidth}
		\textsc{Good to know}:\\
		Any C/C++ source put into this folder (along with the proper \texttt{Android.mk} configuration file), will be compiled by Android's building system and put into device's \texttt{/system/bin} folder, so that will be ready to run within the environment.
    \end{minipage}%
  }%
  \vskip\baselineskip%
	\item \texttt{p2012-common/framework}: contains two main folders, one includes the whole JNI-related stuff (both C/C++ and Java sources), and the other one comprises the Java source of our specific-service Manager class (which handles the Binding procedure, once created) and the \texttt{.aidl} file, which basically is an interface.
	\item \texttt{p2012-common/lib}: contains the C/C++ source code for our specific libraries and headers.
\end{itemize}
Once we have this basic structure, we can go deeper into details of the configuration files needed by the \textit{make system}.
\subsection{Configuration files}
We are mainly operating inside the \texttt{p2012} folder so, unless specified, this will be our working directory for this section.
\begin{itemize}
\item At first, we create a file named \texttt{vendorsetup.sh}, which will contain the name of the \textit{lunch-combo} we want to associate to our new device. My file would look like this:
\begin{verbatim}
	add_lunch_combo full_bosp_p2012-eng
\end{verbatim}
where \texttt{full} indicates that we're starting, as base - from Android build with the most complete set of Application/Services, \texttt{bosp\_p2012} indicates our customised build, \texttt{eng} indicates the development configuration with additional debugging tools.\\
From within the main \texttt{aosp} directory, if you open a shell and launch the following
\begin{verbatim}
	$. build/envsetup.sh
\end{verbatim}
you should see the system including the new configuration.
\item The file \texttt{AndroidProducts.mk} contains the reference to the specific product build makefiles, therefore we indicate here this:
\begin{verbatim}
	PRODUCT_MAKEFILES := $(LOCAL_DIR)/full_p2012.mk
\end{verbatim}
\item And here a piece of our \texttt{full\_p2012.mk} file
\begin{verbatim}
	$(call inherit-product, $(SRC_TARGET_DIR)/product/
	                        languages_full.mk)
	$(call inherit-product, $(SRC_TARGET_DIR)/product/full_base.mk)
	PRODUCT_NAME := full_p2012
	PRODUCT_DEVICE := p2012
	PRODUCT_MODEL := Full BBQUE P2012 Image for Emulator
	# Include the common definition and packages
	include $(LOCAL_PATH)/../p2012-common/bosp_p2012.mk
	include $(call all-makefiles-under,$(LOCAL_PATH))
\end{verbatim}
To our scope, to have a more lightweight build, we change \texttt{languages\_full.mk} and \texttt{full\_base.mk} respectively, with \texttt{languages\_small.mk} and \texttt{generic.mk}.\\
More "flavors" can be found into
\begin{verbatim}
	/aosp/build/target/product/
\end{verbatim}

\end{itemize}