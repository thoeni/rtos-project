\documentclass[acmplain,acmnow]{acmtrans2m}

%\newtheorem{theorem}{Theorem}[section]
%\newtheorem{conjecture}[theorem]{Conjecture}
%\newtheorem{corollary}[theorem]{Corollary}
%\newtheorem{proposition}[theorem]{Proposition}
%\newtheorem{lemma}[theorem]{Lemma}
%\newdef{definition}[theorem]{Definition}
%\newdef{remark}[theorem]{Remark}

\usepackage[dvips]{color}
\usepackage{graphicx}

\title{Exploring Android Stack by implementing a complete flow, from Java application layer to native library}
            
\author{ANTONIO TROINA\\Politecnico di Milano}
            
\begin{abstract} 
This paper aims to describe, through a convienient and properly documented example, the necessary steps to break into the Android stack, and implement a customised native daemon - which relies on its own native library - that can communicate with a Java application (and the other way around) through the Java Native Interface.\\
The first part of this work will cover the Android stack, with a general overview of its main components, while the second part will propose a basic implementation of a client-server paradigm, whose actors will be a Java-coded app (client) and a C-coded daemon (server).
\end{abstract}
            
\category{D.4.7}{Real-time systems and embedded systems}{Android}
\keywords{Android, JNI}
            
\begin{document}
\maketitle

%\newpage
%\thispagestyle{empty}
%\mbox{}

\markboth{Exploring Android Stack}{Introduction}

\section{Introduction}
In 2011, 48.8\% of the smartphones on the market, ran Android OS, contributing to an annual growth of 244\%.\footnote{Source: Canalys. February 2012} This is what an average customer can easily guess by himself. What not everyone can figure, unless he's working in this very specific field, is that lately Android OS - and therefore its open source project AOSP - aroused the curiosity of many embedded developers.\\
As we can see in Fig. \ref{fig:stack} \cite{gargenta}, Andrdoid's architecture can be divided into many components, which we'll briefly analyse - following a bottom-up approach - before going deeper into the core of this work.
\begin{figure}[!htb]
	\centering
	\includegraphics[scale=.6]{images/stack.pdf}
	\caption{Android's stack}
	\label{fig:stack}
\end{figure}
\begin{itemize}
\item \textbf{Linux Kernel}\\
Android kernels are, in sum, forks from the mainline kernel, relying on several custom 	functionalities that are significantly different from what is found in the "vanilla" kernel. Follows a brief enumeration of the most remarkable customizations made to the kernel to satisfy special Android's needs.
	\begin{itemize}
	\item \textsc{Wakelocks}\\
	Instead of letting the system be put to sleep at the user's behest, an "Androidized"
kernel is made to go to sleep as soon and as often as possible. \textit{Wakelocks} are provided to keep the system awake while performing specific processing. The wakelocks and early suspend functionality are actually built on top of Linux's existing power management functionality.
	\item \textsc{Low Memory Killer (LMK)}\\
	Since Android system tipically runs in low-memory conditions, the \textit{out-of-memory} (OOM) handling is crucial: therefore the Android development team has introduced an additional \textit{LMK} - based on priorities to identify \textit{to-be-killed} candidates - that kicks in just before the default kernel OOM killer does (thus preempting its intervention, which occurs only when no memory is left).
	\item \textsc{Binder}\\
	Binder allows apps to talk the System Server and it's what apps use to talk to each others' service components, although they actually perform this communication through  interfaces and stubs generated by the \textit{aidl} tool. In Fig. \ref{fig:binder_flow} and example of the communication flow: the Activity (which is what we're used to call "App", despite this word in the Android Framework acquires different meanings) requires to the \textit{Service Manager} an instance of a specific Service she wants to interact with, and performs this operation through \textit{Stubs} and \textit{Binder}.
		\begin{figure}[!htb]
			\centering
			\includegraphics[scale=.4]{images/binder_flow.pdf}
			\caption{Communication Activity-Service through Binder}
			\label{fig:binder_flow}
		\end{figure}
	\item \textsc{Anonymous Shared Memory}\\
	Typically, a first process creates a shared memory region using \textit{ashmem} and uses Binder to share the corresponding file descriptor with other processes with which it wishes to share the region. \textit{Ashmem} destroys memory regions when all processes referring to them have exited and will shrink mapped regions if the system is in need of memory.
	\item \textsc{Alarm}\\
	Android's alarm driver is actually layered on top of the kernel's existing Real-Time Clock (RTC) and High-Resolution Timers (HRT) functionalities. Android's alarm driver cleverly combines the best of both worlds: by default the driver uses the kernel's High-Resolution Timer (HRT) functionality, but whenever the system is about to suspend itself, it programs the RTC so that the system gets woken up at the appropriate time.
	\item \textsc{Logger}\\
	Logger
	\end{itemize}
\end{itemize}
\markright{Android Internals}
\section{Android Internals - Overview}
\label{overview}
As we can see in Fig.\ref{fig:stack} \cite{remixingand}, Android's architecture can be divided into many components, which we'll briefly analyse - following a bottom-up approach - before going deeper into the core of this work.
\begin{figure}[!htb]
	\centering
	\includegraphics[scale=.6]{images/stack.pdf}
	\caption{Android's stack}
	\label{fig:stack}
\end{figure}
\begin{itemize}
\item \textbf{Linux Kernel}\\
Android kernels are, in sum, forks from the mainline kernel, relying on several custom 	functionalities that are significantly different from what is found in the "vanilla" kernel. Follows a brief enumeration of the most remarkable customizations made to the kernel to satisfy Android's specific needs.
	\begin{itemize}
	\item \textsc{Wakelocks}\\
	Instead of letting the system be put to sleep at the user's behest, an "Androidized"
kernel is made to go to sleep as soon and as often as possible. \textit{Wakelocks} are provided to keep the system awake while performing specific processing. The wakelocks and early suspend functionality are actually built on top of Linux's existing power management functionality.
	\item \textsc{Low Memory Killer (Viking Killer)}\\
	Since Android system typically runs in low-memory conditions, the \textit{out-of-memory} (OOM) handling is crucial: therefore the Android development team has introduced an additional \textit{LMK} - based on priorities to identify \textit{to-be-killed} candidates - that kicks in just before the default kernel OOM killer does (thus preempting its intervention, which occurs only when no memory is left).
	\item \textsc{Binder}\\
	Binder allows apps to talk to the System Server and it's what apps use to talk to each others' service components, although they actually perform this communication through  interfaces and stubs generated by the \textit{aidl} tool. In Fig.\ref{fig:binder_flow}, an example of the communication flow: the Activity (which is what we're used to call "App", despite this word in the Android Framework acquires different meanings) requires to the \textit{Service Manager} an instance of a specific Service she wants to interact with, and performs this operation through \textit{Stubs} and \textit{Binder}.
		\begin{figure}[!htb]
			\centering
			\includegraphics[scale=.4]{images/binder_flow.pdf}
			\caption{Communication Activity-Service through Binder}
			\label{fig:binder_flow}
		\end{figure}
	\item \textsc{Anonymous Shared Memory}\\
	Typically, a first process creates a shared memory region using \textit{ashmem} and uses Binder to share the corresponding file descriptor with other processes which it wishes to share the region with. \textit{Ashmem} destroys memory regions when all processes referring to them have exited and will shrink mapped regions if the system is in need of memory.
	\item \textsc{Alarm}\\
	Android's alarm driver is actually layered on top of the kernel's existing Real-Time Clock (RTC) and High-Resolution Timers (HRT) functionalities. Android's alarm driver cleverly combines the best of both worlds: by default the driver uses the kernel's High-Resolution Timer (HRT) functionality, but whenever the system is about to suspend itself, it programs the RTC so that the system gets woken up at the appropriate time.
	\item \textsc{Logger}\\
	Android defines its own logging mechanisms based on the Android logger driver added to the kernel. The lightweight Android Logger manages a handful of separate kernel-hosted buffers for logging data coming from user-space. The driver maintains circular buffers where it logs every incoming event and returns immediately back to the caller.
	\end{itemize}
So far, the most evident kernel customisations have been described, despite they are actually more, nonetheless a deeper analysis of the Android kernel is out of the scope of this document.
\item \textbf{Hardware Abstraction Layer [HAL]}\\
Keeping in mind Fig.\ref{fig:stack}, let's examine the HAL module. The Android stack typically relies on shared libraries provided by manufacturers to interact with hardware. Actually, Android relies on what can be considered a Hardware Abstraction Layer (HAL), and we can mainly identify the following advantages:
	\begin{itemize}
		\item Separates Android platform logic from specific hardware interfaces
		\item User space HAL offers standard "driver" definitions for graphics, audio, camera, bluetooth, GPS, radio (RIL), WiFi, etc.
		\item Makes porting easier
		\item Other - minor - license-related reasons
	\end{itemize}
\begin{figure}[!htb]
	\centering
	\includegraphics[scale=.25]{images/hal.pdf}
	\caption{Hardware Abstraction Layer structure}
	\label{fig:hal}
\end{figure}
A typical example of the way in which Android abstracts and supports hardware can be seen in Fig.\ref{fig:hal}: it's easy to notice that the hardware support outside the bare Kernel is substantial. As a consequence, Android, to run on a specific hardware, requires more than just a proper driver: a hardware module (not a kernel module) which conforms to the API specified for that type of hardware, must be provided.
\item \textbf{Native Libraries}\\
Android relies on slightly more than a hundred dynamically-loaded libraries (in our testing configuration, 124), which are stored in /system/lib. The largest part of them is actually generated within the AOSP, and a portion comes from external projects, merged into AOSP codebase to add some specific functionalities.
\item \textbf{Native Daemons}\\
As already seen into Linux, Android typically runs some \textit{Daemons}, which are native processes - launched at startup phase, or when expressly requested - that continue to run throughout the lifetime of the system.
\item \textbf{Init}\\
The list of automatically launched daemons during the startup phase (with permission and other options) can be found in \textit{init.rc} file: \textit{\textbf{init}} - as happens into Linux - is the first process started immediately after the Linux boot, and \textit{init.rc} is its configuration file.
\item \textbf{Dalvik VM}\\
Long story short: Dalvik VM is Android's Java Virtual Machine, with a substantial difference, as shown in Fig.\ref{fig:dalvik}, and after written.\\
\begin{figure}[!htb]
	\centering
	\includegraphics[scale=.7]{images/dalvik.pdf}
	\caption{Comparison Java - Dalvik}
	\label{fig:dalvik}
\end{figure}
While in Java the source code is compiled into Java byte code, which can run on the Java VM, in Android there's one more halfway step, which passes through the \textit{dx utility}, which is a \textit{Dex compiler}: \textit{.dex} files, created by post-processing the \textit{.class} files generated by the Java compiler - when uncompressed are 50\% smaller than their originating .jar files - can finally run on the Dalvik VM which therefore can't straight run Java byte code.
\definecolor{lightgray}{gray}{0.93}
  \vskip\baselineskip%
  \par\noindent\colorbox{lightgray}{%
    \begin{minipage}{0.953\textwidth}
		\textsc{Good to know}:\\
		In 2010 an interesting feature of Dalvik has been introduced: a Just-In-Time (JIT) compiler for ARM. This compiler allows \textit{.dex} byte code files to be converted to binary assembly instructions that run natively on the target's CPU (only ARM so far) instead of being interpreted one instruction at a time by the VM. The first time you load the app it may take a bit longer, but once an App has been \textit{"JIT-ed"}, then it loads and runs much faster than being interpreted.
    \end{minipage}%
  }%
  \vskip\baselineskip%
\end{itemize}
Before we go on analysing the \textit{System Services} block (again, Fig.\ref{fig:stack}), it's worth focusing on JNI interface, which basically permits the communication between the \textit{Java} world and the \textit{Native} one.
\begin{itemize}
\item \textbf{Java Native Interface [JNI]}\\
Java written code sometimes needs to interface to native code written in C/C++, and this sentence comes especially true in the embedded systems' world: in fact, to access to low-level functionalities, Java applications need to interact with native daemons and libraries. This bridge role is played by what is known as \textit{Java Native Interface}.\\
JNI allows programmers to take advantage of the power of the Java platform, without having to abandon their investments in legacy code: this means also that you can port your C/C++ code to Android with a bit of investment on implementing a sort of \textit{"glue code"} which lets Android applications using it. JNI has been defined as \textit{"a dark art reserved to the initiated"} \cite{embandroid} therefore, for the time being, it's enough to have an idea of what it is.
\item \textbf{System Services}\\
System services (e.g. Location service, Sensor service, WiFi service, Alarm service, Telephony service, Bluetooth service, and so on) are always on, always running, and readily available for developers to tap into, started at boot time and are guaranteed to be running by the time your application launches. They cooperate together through Binder, the mechanism on which all system services are built.\\
System services are basically divided in two parts: mainly Java-coded services (but two), running under the \textit{system server} process, and C/C++ coded services, running under the \textit{media server} process.\\
As we did for JNI, we won't go deeper into the details of System Services now, although we'll better understand its functioning through an example. Fully understanding the internals of Android's system services has been defined as \textit{trying to swallow a whale} \cite{embandroid}.
\item \textbf{Service Manager}\\
As the director of orchestra, the Service Manager coordinates all the system services, and allows Android Applications to access System Services: as we'll see in the second part of this work, an Application can invoke the
\begin{verbatim}
	serviceManager.getSystemService(SERVICE_NAME);
\end{verbatim} method on the ServiceManager object. This interaction nevertheless isn't as simple as it could seem at first sight, since the whole process uses the Binder paradigm. (Figg. \ref{fig:service_manager} and \ref{fig:binder_flow}) despite Android Development Team made all this quite transparent to the Application developer, who typically doesn't need to implement his own service, but just to use the ones already built by AOSP.
\begin{figure}[!htb]
	\centering
	\includegraphics[scale=.5]{images/service_manager.pdf}
	\caption{Interaction "Application - Service manager", through Binder}
	\label{fig:service_manager}
\end{figure}
\item \textbf{Android Framework Libraries}\\
Libraries that allow the Application, through Java Interfaces, to access the proper native libraries: whenever an instance of a library class is created, a call to the ServiceManager - through binder - is made, to get the proper Service (or System Service) as Interface (binded to an .aidl file), which finally permits the Application to call Java libraries' methods that, through JNI, are connected to the native library.
\end{itemize}
The higher level isn't that much of our interest for this work, as it includes the "Apps" layer, which actually are called "Activities", and they are just the presentation level of the whole stack. Nevertheless we will see in the last part of this work (Section \ref{appLayer} and Fig.\ref{fig:screenshot}) the implementation of a very simple Activity to graphically interact with our custom native daemon.
\markright{Environment setup}
\section{Environment setup}
\label{setup}
Basically, to have a foundation we can work on, a ready-to-run AOSP system should be setup and properly configured. The most important instructions can be found at the AOSP project website: \texttt{http://source.android.com/}\\
In this section the basic steps will be presented, especially highlighting the main issues that have been encountered during the development.
\subsection{Initialize the build environment}
The on-line guide to initialize the system can be found at:\\
\texttt{http://source.android.com/source/initializing.html}\\
It is based on Ubuntu LTS (10.04), nevertheless the environment I used was Linux Mint 13, which is based on Ubuntu (12.04).\\
Leaving aside the default recommendations by the AOSP webpage, let's see what actually went wrong, and how to fix it:
\begin{itemize}
\item \textbf{Java Development Kit}\\
The android source suggests to install, among the others, the OpenJDK package:
	\begin{verbatim}
		$ sudo apt-get install openjdk-6-jdk
	\end{verbatim}
This didn't work, since I found out that Android 2.3+ requires Java 6 to build correctly. While OpenJDK did work for 2.3, it doesn't for Android 4.0: you need Sun Java Development Kit. A solution is to download the JDK 1.6 binary release from Sun's Java download site and unzip it somewhere in your system, for example:
	\begin{verbatim}
		$ sudo mkdir -p /opt/java/64/
		$ sudo cp jdk-6u33-linux-x64.bin /opt/java/64
		$ sudo su -
		$ cd /opt/java/64
		$ chmod +x jdk-6u29-linux-x64.bin
		$ ./jdk-6u33-linux-x64.bin
		$ exit
	\end{verbatim}
Add the new Java to your shell environment's \texttt{\$PATH} so it takes priority over other Javas you might have installed (yes, this means that different JDK can coexist):
	\begin{verbatim}
		$ echo 'export PATH=/opt/java/64/jdk1.6.0_33/bin:$PATH' \
		       >> ~/.bashrc
	\end{verbatim}
If you relaunch your terminal (do it) you should be able to see the change with the command \texttt{java -version}
\item \textbf{GCC - The Gnu Compiler Collection}\\
By default I found on my system the latest version of the \texttt{gcc} compiler, which - at the time of this writing is the \texttt{4.6}: this version works smoothly with Android Jelly Bean, but presented some fatal errors which didn't let me complete the compilation on Icecream Sandwich. A solution - for the latter - has been to install the \texttt{4.4} version (as for JDK, two different versions can coexist on the same system).
	\begin{verbatim}
		$ sudo apt-get install gcc-4.4 g++-4.4 \
		                       g++-4.4-multilib gcc-4.4-multilib
	\end{verbatim}
This works, with the watchfulness of explicitly declare you want to use this specific gcc version at compile time, by passing it as argument to the make command, as follows:
\definecolor{lightgray}{gray}{0.93}
  \vskip\baselineskip%
  \par\noindent\colorbox{lightgray}{%
    \begin{minipage}{0.953\textwidth}
		\textsc{Remember}:\\
			\texttt{\$ make CC=gcc-4.4 CXX=g++-4.4 -j8}
    \end{minipage}%
  }%
  \vskip\baselineskip%
I put stress on this because I'm sure it can save you time, and prevent you from banging the desktop with your forehead, in case you're building an older version of Android.\\
The \texttt{-j8} option tells the compiler to launch 8 parallel threads, therefore if you are compiling on a multi-core architecture, this may save you quite a lot of time: you can choose the value that better fits your cpu architecture.
\item \textbf{Libraries}\\
Since I built and installed other packages during my work, I experienced that the most commonly replaced ones were:
	\begin{itemize}
		\item libncurses5-dev:i386
		\item libgl1-mesa-glx:i386\\
		For this one is convenient to do this:
		\begin{verbatim}
			$ sudo ln -s /usr/lib/i386-linux-gnu/mesa/libGL.so.1 \
			             /usr/lib/i386-linux-gnu/libGL.so
		\end{verbatim}
	\end{itemize}
Therefore, if the compilation - that previously was working - suddenly doesn't work anymore, check whether you still have those two libraries and their versions.
\end{itemize}
Last suggestion, as the AOSP website says, it's recommended enabling the \texttt{CCACHE} option, to save time during the following compilations. You can do it by executing the following bash command:
\begin{verbatim}
$ export USE_CCACHE=1
\end{verbatim}
\subsection{Downloading the source tree}
All instructions to download the source files can be found at this address:\\
\texttt{http://source.android.com/source/downloading.html}
\subsection{Building the system}
All instructions to build the system can be found at this address:\\
\texttt{http://source.android.com/source/building.html}
\markright{Customising}
\section{Customising}
\label{Customising}
This work, as stated in the abstract, aims to implement a native daemon (constantly running) which will be exposed as a System Service to the Java application layer. The Java application will be able to communicate with the daemon through JNI and the native library, and the other way around will be implemented by using a callback function.\\
Since this project is focused on understanding the basics of Android, to afterwards integrate a native RTRM application as a System Service - therefore accessible by the Java environment - I decided to implement this simple daemon as a dummy-RTRM which will communicate with the application the number of available "cores" (actually will be just an \texttt{int}) and allow it to "require" or "release" cores (with a sort of \texttt{core++} and \texttt{core--} functions).
\subsection{Setting up the directory structure}
The followed approach has been to keep separate the main code from our customised code, so that our platform-specific code will be compiled when the user will choose to build our specific system, instead of the default one. This choice results in a more atomic and modular way to operate.\\
In the \texttt{aosp} directory, there's a folder named \texttt{device}, which typically contains folders indicating manifacturers, and into each manifacturer's folder, the specific architecture folder: in our case, I named \texttt{bosp} the manifacturer folder, and \texttt{p2012} the architecture folder. To properly organize files, we have a main \texttt{p2012} folder, and a \texttt{p2012-common} folder, which contains the largest part of our customised code.
\begin{verbatim}
device/
├── bosp/
│   ├── p2012/
│   │   ├── app/
│   ├── p2012-common/
│   │   ├── app/
│   │   ├── bin/
│   │   ├── framework/
│   │   └── lib/
\end{verbatim}
Let's briefly see what these folders will contain:
\begin{itemize}
	\item \texttt{p2012/app}: contains configuration files to identify the board, the architecture, the packages that will be built-in with this version, and the possible "Activities" that will be coded especially for this release.
	\item \texttt{p2012-common/app}: contains \texttt{ServiceApp} that will register our new service to the \textit{Service Manager} as a remote service, through the call
	\begin{verbatim}
		ServiceManager.addService(REMOTE_SERVICE_NAME, this.serviceImpl);
	\end{verbatim}
	\item \texttt{p2012-common/bin}: contains the C/C++ code of the \textit{native daemon}. Once exported as local module, its name will be added to the \texttt{init.rc} file, to let it start at bootup stage.
\end{itemize}

\bibliographystyle{acmtrans}
\bibliography{myrefs}
\nocite{embandroid}
\nocite{andinternals}
\nocite{opersys}
\nocite{learningandroid}
\nocite{jni}

\begin{received}
August, 2012
\end{received}
\end{document}


