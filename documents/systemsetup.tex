\markright{Environment setup}
\section{Environment setup}
\label{setup}
Basically, to have a foundation we can work on, a ready-to-run AOSP system should be setup and properly configured. The most important instructions can be found at the AOSP project website: \texttt{http://source.android.com/}\\
In this section the basic steps will be presented, especially highlighting the main issues that have been encountered during the development.
\subsection{Initialize the build environment}
The on-line guide to initialize the system can be found at:\\
\texttt{http://source.android.com/source/initializing.html}\\
It is based on Ubuntu LTS (10.04), nevertheless the environment I used was Linux Mint 13, which is based on Ubuntu (12.04).\\
Leaving aside the default recommendations by the AOSP webpage, let's see what actually went wrong, and how to fix it:
\begin{itemize}
\item \textbf{Java Development Kit}\\
The android source suggests to install, among the others, the OpenJDK package:
	\begin{verbatim}
		$ sudo apt-get install openjdk-6-jdk
	\end{verbatim}
This didn't work, since I found out that Android 2.3+ requires Java 6 to build correctly. While OpenJDK did work for 2.3, it doesn't for Android 4.0: you need Sun Java Development Kit. A solution is to download the JDK 1.6 binary release from Sun's Java download site and unzip it somewhere in your system, for example:
	\begin{verbatim}
		$ sudo mkdir -p /opt/java/64/
		$ sudo cp jdk-6u33-linux-x64.bin /opt/java/64
		$ sudo su -
		$ cd /opt/java/64
		$ chmod +x jdk-6u29-linux-x64.bin
		$ ./jdk-6u33-linux-x64.bin
		$ exit
	\end{verbatim}
Add the new Java to your shell environment's \texttt{\$PATH} so it takes priority over other Javas you might have installed (yes, this means that different JDK can coexist):
	\begin{verbatim}
		$ echo 'export PATH=/opt/java/64/jdk1.6.0_33/bin:$PATH' \
		       >> ~/.bashrc
	\end{verbatim}
If you relaunch your terminal (do it) you should be able to see the change with the command \texttt{java -version}
\item \textbf{GCC - The Gnu Compiler Collection}\\
By default I found on my system the latest version of the \texttt{gcc} compiler, which - at the time of this writing is the \texttt{4.6}: this version works smoothly with Android Jelly Bean, but presented some fatal errors which didn't let me complete the compilation on Icecream Sandwich. A solution - for the latter - has been to install the \texttt{4.4} version (as for JDK, two different versions can coexist on the same system).
	\begin{verbatim}
		$ sudo apt-get install gcc-4.4 g++-4.4 \
		                       g++-4.4-multilib gcc-4.4-multilib
	\end{verbatim}
This works, with the watchfulness of explicitly declare you want to use this specific gcc version at compile time, by passing it as argument to the make command, as follows:
\definecolor{lightgray}{gray}{0.93}
  \vskip\baselineskip%
  \par\noindent\colorbox{lightgray}{%
    \begin{minipage}{0.953\textwidth}
		\textsc{Remember}:\\
			\texttt{\$ make CC=gcc-4.4 CXX=g++-4.4 -j8}
    \end{minipage}%
  }%
  \vskip\baselineskip%
I put stress on this because I'm sure it can save you time, and prevent you from banging the desktop with your forehead, in case you're building an older version of Android.\\
The \texttt{-j8} option tells the compiler to launch 8 parallel threads, therefore if you are compiling on a multi-core architecture, this may save you quite a lot of time: you can choose the value that better fits your cpu architecture.
\item \textbf{Libraries}\\
Since I built and installed other packages during my work, I experienced that the most commonly replaced ones were:
	\begin{itemize}
		\item libncurses5-dev:i386
		\item libgl1-mesa-glx:i386\\
		For this one is convenient to do this:
		\begin{verbatim}
			$ sudo ln -s /usr/lib/i386-linux-gnu/mesa/libGL.so.1 \
			             /usr/lib/i386-linux-gnu/libGL.so
		\end{verbatim}
	\end{itemize}
Therefore, if the compilation - that previously was working - suddenly doesn't work anymore, check whether you still have those two libraries and their versions.
\end{itemize}
Last suggestion, as the AOSP website says, it's recommended enabling the \texttt{CCACHE} option, to save time during the following compilations. You can do it by executing the following bash command:
\begin{verbatim}
$ export USE_CCACHE=1
\end{verbatim}
\subsection{Downloading the source tree}
All instructions to download the source files can be found at this address:\\
\texttt{http://source.android.com/source/downloading.html}
\subsection{Building the system}
All instructions to build the system can be found at this address:\\
\texttt{http://source.android.com/source/building.html}